\chapter{アンケートによる評価と考察}
\label{chap:result}

本章では、そのアプリについての評価を行い、考察を述べる。

\section{評価結果}
今回は、当研究室のメンバーに対してアプリを実際に利用してもらい、アンケートを行った。
その結果、以下の様な意見を得た。
\begin{itemize}
\item 非常に動きが独特で面白い。
\item 横や縦にスライドして動く描写が楽しく、e-learningコンテンツを見ていく面白さがあって良い。
\item パブリッシュされた一つのアプリとしてリリースするより、HTMLページとしてリリースする方が良い。より一般的に利用して貰えるようになるし、iOSやタブレット端末への移植の手間も省ける。
\item PDFファイルまでアクセスできるのは素晴らしいが、一度ダウンロードが行われてからビューワを起動するため手間がかかるのではないか。PDFファイルをHTMLで直接閲覧できるよう構築できるようにした方が良い。
\item タブレット等の横幅のあるデバイスならば、レイアウトを工夫して、画面遷移なしでブラウザを横に起動させたりできるのではないか。そうすればより良い使いやすさを追求できると考える。
\item 検索ボックスやタイトルの表示部が透過されているため、下のアイコンと被っており見にくい。
\item 検索ボックスが非常にシンプルなため、一見しただけではどのように使うのか分からない。説明を加えてほしい。
\item 現在どの結果に注目しているのかが分かりにくいため、アイコンを拡大したり、印をつけるなどしてほしい。
\item アイコンをタップすれば良いと思ってしまうが、実際には下のタイトル部分をタップしなければならないため、紛らわしい。アイコンから直接リンクできるようにすれば良いのではないか。
\item アイコンが大量に並ぶと、どれが重要なコンテンツなのか判断がつかない。表示数を制限して、より重要な結果に絞ってから表示するべきである。
\item 検索結果で出てきたアイコンは、単なる検索結果だけなので、その中における序列がなく最も有効なコンテンツがわかりづらい。検索結果を深く融合して、まとめてランク付けを行なってほしい。
\item URLからのアイコン画像取得に失敗するエラーがある。アプリ自体の価値が極度に下がってしまうため、なるべく改善した方が良い。
\item 画像ファイルの読み込みが遅い。画像処理専用のサーバ等を使い、ファイルをより早く処理できるようにすれば良いのではないか。
\item 全体的にどれが何を表しているのか分からづらい。ある程度慣れているユーザならば、PDFのマークには見覚えがあるが、初めて見たユーザや子供には、それが何を表しているのか、今ひとつ分からない。随所に説明を入れたり、アドバイス用のメッセージを出すなどしてほしい。
\item カスタマイズ性を重視して、拡張によって検索エンジンを導入できるような機能がほしい。
\item 表示形式を柔軟に変更できるようにしてほしい。画面を回転・ズームし、自分の好きな方向に簡単に移動できる機能があれば、より操作性を高めることができると考える。
\end{itemize}

\section{考察}
今回の解決すべき命題は、
\begin{itemize}
\item Mashupの特性を活かした、e-learningコンテンツ検索エンジン開発
\item Android端末上であることを意識した、ユーザビリティの高いWWW視覚化
\end{itemize}
の2点であった。

\subsection{e-learningコンテンツ検索エンジンとしての性能}
検索結果の一覧性が高いことや、PDFデータ等にダイレクトにアクセスできることを評価する声が一定数あった。この点より、当初の目的としての検索エンジン開発は概ね達成したと言えるだろう。その一方、APIを通じて得た検索結果が、アイコンとしてただ広がるだけのGUIに不満があったことは否めず、よりシンプルな、ひと目で分かる検索結果を返すための仕組みを、Mashupを通じて再度構成するべきだろう。

\subsection{Android端末におけるWWW視覚化とユーザビリティ}
スワイプ動作の独特な操作性や、レンダリングがスムーズで心地良く、面白いとの意見を多数得られた。この点より、Android端末としての利点は活かし、ユーザエクスペリエンスの点ではある程度の評価を得られたと考える。しかし、ユーザビリティの高さ、という部分が完全におざなりになっている面に関しては、意見の数々より自明のことであろう。また、納豆ビュー\cite{natto}に実装されていた、z軸方向における回転は実現できたものの、x-y方向への一意的な意味付け、回転操作等は実現できず、三次元空間における操作性の点に関しても、幾つかの課題が残った。より自由な操作性を実現するため、ローテートやダブルタップなど、より複雑な操作によって実現できることも検討するべきだろう。