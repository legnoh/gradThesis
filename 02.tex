\chapter{MashupとWebAPI}
\label{chap:webapi}
本章では、表題となっているMashupと呼ばれる開発手法に加え、WebAPIと呼ばれるタイプのAPIについて解説する。
\section{Mashupとは}
Mashupとは、2つ以上のWebAPIを組み合わせて1つのWebサービスやアプリケーションを構成する手法のことである。元来、利用価値の高いWebサービスを作るためには、独自に検索エンジンや結果応答用のサーバを構築する必要があり、目的のWebサービスを作るために多大な努力をする必要があった。しかし、Mashupでは、既存のWebサービスを組み合わせることにより、短期間で価値の高いWebサービスを製作することができる。
\section{WebAPIとは}
WebAPIとは、インターネットを介して利用することのできるアプリケーション・プログラミング・インターフェイス(API)のことである。殆どのWebAPIが一般的なURLの形式を取っており、HTTPによるPOSTメソッドを用いて、パラメータを付加したURLを使用してアクセスしてデータを取得する。返ってくるデータはXML、JSONのどちらかが一般的である。今回用いるWebAPIは、以下の4つである。
\subsection{Yahoo!検索Web API-ウェブ検索API\footnote{ウェブ検索APIは、2013年3月頃を目処にAPIのリクエストURLが変更される予定であり、これはそれまで公開されていたアップグレード版ウェブ検索APIを使用している。}}
\begin{tabular}{c|l}
開発 & ヤフー株式会社\\
URL & http://search.yahooapis.jp/PremiumWebSearchService/V1/webSearch\\
機能 & Web上に公開されているページを検索する
\end{tabular}
\subsection{Yahoo!検索Web API-画像検索API\footnote{画像検索APIは、2013年3月頃を目処にAPIのリクエストURLが変更される予定であり、これはそれまで公開されていたアップグレード版画像検索APIを使用している。}}
\begin{tabular}{c|l}
開発 & ヤフー株式会社\\
URL & http://search.yahooapis.jp/PremiumImageSearchService/V1/imageSearch\\
機能 & Web上に公開されている画像を検索する
\end{tabular}
\subsection{Youtube Data API}
\begin{tabular}{c|l}
開発 & Google Inc.\\
URL & http://gdata.youtube.com/feeds/api/videos\\
機能 & Youtubeの機能(動画の検索、アップロード、再生リストの作成など)を利用する
\end{tabular}
\subsection{Product Advertising API}
\begin{tabular}{c|l}
開発 & Amazon.com, Inc.\\
URL & http://ecs.amazonaws.jp/onca/xml\\
機能 & Amazon の商品情報や関連コンテンツを検索する
\end{tabular}