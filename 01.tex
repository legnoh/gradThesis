\chapter{序論}
\label{chap:introduction}

本章では、本研究の背景、それを踏まえた上での研究の目標・目的、そして文書の構成について述べる。

\section{背景}

2007年、iOS、Android OSの両オペレーティング・システムを搭載したタブレット端末やスマートフォン端末が発表された。
これら端末はスペック的にそれほど高くものではないものの、タッチパネルの搭載による直感的な操作、携帯性の高さ、Wi-fi接続によるインターネット接続が可能といった多数のメリットを兼ね備えており、欧米を中心に今日まで爆発的に普及してきている。\footnote{株式会社シード・プランニングの行った2012年7月の市場調査\cite{smartphoneresearch}によると、日本でのスマートフォン普及率は40\%前後と先進国の中ではやや低調である。元々高品質な携帯電話が普及しており、プラットフォームが盤石であったことが要因であると考えられている。}\\
一方、e-learningとは、パーソナルコンピュータなどの情報機器を用いて行う学習のことである。1990年代後半からのPCの普及と共に様々な分野で用いられるようになり、現在ではe-learningのコンテンツ共有を目的とした規格\cite{scorm}や大学設置基準に基づく文部科学省告示の中にe-learningに関する項目が記述される\cite{monkasho}など、制度や規格も整備されたものとなっている。\\
e-learningサイトの多くは、その基盤がタブレット端末、スマートフォン端末が発表されるより前に発表されたものであり、

iTunes U


\section{本文書の構成}

第1章の最後は、文書全体の構成を大まかに書くとよいらしい。

第\ref{chap:introduction}章では本テンプレートの概要みたいなものを書いた。第\ref{chap:howto}章では、本テンプレートの使い方を説明する。第\ref{chap:latex}章で図表や数式の挿入など代表的な\LaTeX コマンドを解説する。第\ref{chap:conclusion}章では、『序論』で始めたら『結論』で終われと書いた手前書かざるを得ないので、なにか結論らしいことを書く。付録として、テンプレートのサンプルになるように無理矢理ゴミを添付する。