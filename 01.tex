\chapter{序論}
\label{chap:introduction}

本章では、本研究の背景、それを踏まえた上での研究の目標・目的、そして文書の構成について述べる。

\section{背景}

2007年、iOS、Android OSの両オペレーティング・システムを搭載したタブレット端末やスマートフォン端末が発表された。
これら端末はスペック的にそれほど高くものではないものの、タッチパネルの搭載による直感的な操作、携帯性の高さ、Wi-fi接続によるインターネット接続が可能といった多数のメリットを兼ね備えており、欧米を中心に今日まで爆発的に普及してきている。\footnote{株式会社シード・プランニングの行った2012年7月の市場調査\cite{smartphoneresearch}によると、日本でのスマートフォン普及率は40\%前後と先進国の中ではやや低調である。元々高品質な携帯電話が普及しており、プラットフォームが盤石であったことが要因であると考えられている。}\\
一方、e-learningとは、パーソナルコンピュータなどの情報機器を用いて行う学習のことである。1990年代後半からのPCの普及と共に様々な分野で用いられるようになり、現在ではe-learningのコンテンツ共有を目的とした規格\cite{scorm}や大学設置基準に基づく文部科学省告示の中にe-learningに関する項目が記述される\cite{monkasho}など、制度や規格も整備されたものとなっている。\\
だがe-learningコンテンツを提供するサイトは、その多くがタブレット端末、スマートフォン端末が発表されるより前に製作されたものである。現状、スマートフォンやタブレットからe-learningコンテンツに対してダイレクトにアクセスするためには、まずはうまくコンテンツだけがヒットするような副次的な検索キーワードを考えて、PC用のサイトから小さなボタンをタップし、コンテンツをダウンロードし、更にテキストや動画で別々の検索エンジンを使わなければならない、といったようにかなりの労力を要する。\footnote{iOSについては、e-learning用にユーザーインターフェースが最適化されたiTunesU\cite{itunesu}が存在するが、これはiTunesStore内にあるコンテンツのみを対象としており、WWW上に存在するコンテンツをすべて検索対象とすることはできない}。\\
当研究では、こういった問題点を改良するため、スマートフォン端末・タブレット端末上でe-learningコンテンツを簡単に検索し、自在かつ直感的に閲覧、ダウンロードできるアプリの開発を行う。


\section{本文書の構成}
この第\ref{chap:introduction}章では、本論文を書くに至った背景とその構成を説明している。\\
第\ref{chap:webapi}章では、検索エンジンの構成に使用したWebAPIと、それらを統合した手法について説明する。第\ref{chap:search}章でキーワードを用いた精度向上手法について説明する。第\ref{chap:visualize}章では、WWW視覚化という目線での先行研究や開発事例、そして解決方法についての案を提起する。第\ref{chap:coding}章では、アプリ開発のための言語や手法についての詳細を概説する。第\ref{chap:ledoxea}章では、開発したアプリの利用方法と、その特徴について説明する。第\ref{chap:result}章では、そのアプリについての評価を行い、考察を述べる。第\ref{chap:conclusion}章では、本研究のまとめを行い、今後の課題を列挙する。