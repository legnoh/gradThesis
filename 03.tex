\chapter{検索エンジンの精度向上}
\label{chap:search}

この章では、キーワードを用いた検索結果の精度向上手法について説明する。

\section{先行研究}
当研究室にて行われた研究として、以下の研究がある。

\subsection{e-learningコンテンツにおけるドキュメントサーチの最適化\cite{docsearch}}
Yahoo!Search BOSS APIを用いて、有用なWebページへのヒット率を向上させる実験が行われた。その結果、検索を行う主なキーワードとは別に、HTMLページ中に埋め込まれた<meta>タグのkeyword属性のパラメータとして特に多いものをサブキーワードとして検索を行う方法にて、有用なWebサイトへのヒット率が向上することが確認された。

\section{サブキーワードの選出}
上記研究結果から、e-learningコンテンツへのアクセス精度を高めるため、サブキーワードを選出する方法を考える。

\subsection{選出過程}
サブキーワードを選出する方法に上記の先行研究結果を用いることを試みたのだが、HTMLをキャッシュせず、検索を行うたびに毎回数十ページへアクセスを行なっていたため、タブレット端末やスマートフォン端末など、通信が不安定になる可能性が高い端末でこれを用いることは難しいと判断した。よって今回は、予めヒット率が向上すると考えられるキーワードを適当に予測、検証し、1〜2個程度のサブキーワードを選出した。

\subsection{選出結果}
\begin{itemize}
\item 基礎
\item 講座
\end{itemize}
以上の2個をサブキーワードとする。サブキーワードはORに設定して検索を行い、どちらか片方がヒットした場合の結果を引き出すようにする。