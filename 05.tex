\chapter{開発手法}
\label{chap:coding}

本章では、本研究の背景、それを踏まえた上での研究の目標・目的、そして文書の構成について述べる。

\section{概観}
タブレットやスマートフォンアプリで使える言語としては、JavaやObjective-Cが挙げられる。しかし、これらの言語で作る3次元CGはOpenGLを用いるものが多いため、全体的に難易度が高く、一度コードを書いてしまうと移植性も低い。その点を考慮した結果、今回はADOBE AIR(ActionScript)上で、Away3Dをライブラリとして採用する方針を取った。

\section{Away3D}
Away3Dについて、基本的な構造を解説する。

\subsection{Away3Dとは}
Away3D\cite{away3d}とは、Stage3Dと呼ばれるADOBE AIRからGPUを利用するAPI上で機能する、3D描画ライブラリである。基本的な3Dの描画機能に加え、プリミティブ数やエフェクトの種類が豊富に揃っており、3次元CGの一通りの機能を利用することができる。Apache 2.0ライセンスにて無償で配布されている。

\subsection{Android上への移植}
また、Away3Dの特徴として、ADOBE AIRがマルチプラットフォームに対応しているため、端末を選ばずに移植が容易であることが挙げられる。今回は開発の都合上、Androidスマートフォン上での設計を行なっているが、コードを一部手直しするだけで、iOSやPC(ブラウザ)上でも利用可能になるよう設計を行った。