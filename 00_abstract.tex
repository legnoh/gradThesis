% ■ 概要の出力 ■
%		begin{jabstract}〜end{jabstract}	:日本語の概要
%		begin{eabstract}〜end{eabstract}	:英語の概要
%		※ 不要ならばコマンドごと消せば出力されない。

% 日本語の概要
\begin{jabstract}
Mashupはポータルと同様にコンテンツ検索を可能にするツールであるが、Web2.0を用いてコンテンツを分類する、異なるサイトのAPIを利用する、クライアント・サーバー双方で検索ができる、検索は任意の構造化ハイブリッドコンテンツとして個々のコンテンツを混合して行える、XML変換のREST,RSS,Atom等が利用等が異なる。この特徴を利用し、既提案モデル納豆ビューによるコンテンツを三次元空間にグラフとして展開し操作可能とするAndroid上での検索システムを構築した。
\end{jabstract}

% 英語の概要
\begin{eabstract}
Mashup enables us to search contents like a portal site. However, there is difference to classify contents with Web2.0, use webAPI in external sites, search in the client-side and  the server-side, merge individual contents as a any hybrid content, and use REST, RSS, and Atom with converted XML. This time I developed the search system that make it possible to control the graph in the 3D space by existing proposed model(NATTO View), with these peculiarity on Android devices.
\end{eabstract}

