% ■ 概要の出力 ■
%		begin{jabstract}〜end{jabstract}	:日本語の概要
%		begin{eabstract}〜end{eabstract}	:英語の概要
%		※ 不要ならばコマンドごと消せば出力されない。

% 日本語の概要
\begin{jabstract}
Mashupはポータルと同様にコンテンツ検索を可能にするツールであるが、Web2.0を用いてコンテンツを分類する、異なるサイトのAPIを利用する、クライアント・サーバー双方で検索ができる、検索は任意の構造化ハイブリッドコンテンツとして個々のコンテンツを混合して行える、XML変換のREST、RSS,Atom等が利用等が異なる。この特徴を利用し、既提案モデル納豆ビューによる特徴空間におけるコンテンツの特徴ベクトルに基づく検索システムを構築した。
\end{jabstract}

% 英語の概要
\begin{eabstract}
Tablet and Smartphone (e.g. iOS and Android ) devices today has a fairly, those app's value is appreciated in educational field(e.g. e-learning). However, those app's GUI is inexperienced and not considered carefully. Therefore, it is meaningful to build educational support service. This reseach is development and veritificcation of e-learning search app with mashup in Android.
\end{eabstract}