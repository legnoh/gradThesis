% 独自のコマンド

% ■ アブストラクト
%	\begin{jabstract} 〜 \end{jabstract}	:日本語のアブストラクト
%	\begin{eabstract} 〜 \end{eabstract}	:英語のアブストラクト

% ■ 謝辞
%	\begin{acknowledgment} 〜 \end{acknowledgment}

% ■ 文献リスト
%	\begin{bib}[100] 〜 \end{bib}


\newif\ifjapanese

\japanesetrue	% 論文全体を日本語で書く(英語で書くならコメントアウト)

\ifjapanese
	\documentclass[11pt]{jreport}
	\renewcommand{\bibname}{参考文献}
	\newcommand{\acknowledgmentname}{謝辞}
\else
	\documentclass[11pt]{report}
	\newcommand{\acknowledgmentname}{Acknowledgment}
\fi
\usepackage{ascmac}
\usepackage{graphicx}
\usepackage{multirow}
\usepackage{ylab_thesis}
\DeclareFontShape{JY1}{mc}{m}{it}{<5> <6> <7> <8> <9> <10> sgen*min
    <10.95><12><14.4><17.28><20.74><24.88> min10 <-> min10}{}
\DeclareFontShape{JT1}{mc}{m}{it}{<5> <6> <7> <8> <9> <10> sgen*tmin
    <10.95><12><14.4><17.28><20.74><24.88> tmin10 <-> tmin10}{}
\usepackage{times}
\usepackage[stable]{footmisc}

\bindermode	% バインダ用余白設定

% 日本語情報(必要なら)
\jclass	{卒業論文}							% 論文種別
\jtitle		{Mashupによるe-learningコンテンツ検索表示}			% タイトル。改行する場合は\\を入れる
\juniv		{佐賀大学}						% 大学名
\jfaculty	{理工学部知能情報システム学科}				% 学部、学科
\jauthor	{甲斐 遼馬}						% 著者
\jnumber	{08233014}						% 学籍番号
\jadvisor	{新井 康平}{教授}					% 指導教官、形式は『{名前}{肩書}』
\jchief	{渡邉 義明}{教授}					% 学科長名、形式は『{名前}{肩書}』
\jhyear	{24}								% 平成○年度
\jsyear	{2012}							% 西暦○年度
\jkeyword	{mashup、Android、e-learning}			% 論文のキーワード

% 英語情報(必要なら)
\eclass	{Graduation Thesis}					% 論文種別
\etitle		{The develop of e-learning contents viewer by “Mashup”}	% タイトル。改行する場合は\\を入れる
\euniv	{Saga University}						% 大学名
\efaculty	{Department of Information Science, Faculty of Science and Engineering, Saga University}	% 学部、学科
\eauthor	{Ryoma KAI}					% 著者
\enumber	{08233014}						% 学籍番号
\eadvisor	{Professor}{Kohei ARAI}				% 指導教官、形式は『{肩書}{名前}』
\echief		{Professor}{Yoshiaki WATANABE}				% 学科長名、形式は『{肩書}{名前}』
\eyear	{2012}							% 西暦○年
\ekeyword	{Mashup,Android,e-learning}		% 論文のキーワード





\begin{document}

\jmaketitle		% 表紙(日本語)、不要ならコメントアウト
\emaketitle		% 表紙(英語\ref{chap:latex})、不要ならコメントアウト

% ■ 概要の出力 ■
%		begin{jabstract}〜end{jabstract}	:日本語の概要
%		begin{eabstract}〜end{eabstract}	:英語の概要
%		※ 不要ならばコマンドごと消せば出力されない。

% 日本語の概要
\begin{jabstract}
近年、iOSやAndroidなどのタブレット端末・スマートフォン端末の普及が進み、それらの教育目的での利用価値も俄然高く評価されてきている。しかし、教育方面での利用のためのGUIは未だに未熟、またはなかなか考慮されにくいのが現状であり、タブレット端末での効果的な学習を支援するサービスを作ることは非常に意義深いことであると考える。本研究では、タブレット端末におけるe-learning検索アプリを、mashupと呼ばれる開発手法を用いて柔軟に開発した後、検証を行ったものである。
\end{jabstract}

% 英語の概要
\begin{eabstract}
Tablet and Smartphone (e.g. iOS and Android ) devices today has a fairly, those app's value is appreciated in educational field(e.g. e-learning). However, those app's GUI is inexperienced and not considered carefully. Therefore, it is meaningful to build educational support service. This reseach is development and veritificcation of e-learning search app with mashup in Android.
\end{eabstract}	% アブストラクト。要独自コマンド、include先参照のこと

\tableofcontents	% 目次
% \listoffigures		% 表目次、不要ならコメントアウト
% \listoftables		% 図目次、不要ならコメントアウト

\pagenumbering{arabic}

\chapter{序論}
\label{chap:introduction}

本章では、本研究の背景、それを踏まえた上での研究の目標・目的、そして文書の構成について述べる。

\section{背景}

2007年、iOS、Android OSの両オペレーティング・システムを搭載したタブレット端末やスマートフォン端末が発表された。
これら端末はスペック的にそれほど高くものではないものの、タッチパネルの搭載による直感的な操作、携帯性の高さ、Wi-fi接続によるインターネット接続が可能といった多数のメリットを兼ね備えており、欧米を中心に今日まで爆発的に普及してきている。\footnote{株式会社シード・プランニングの行った2012年7月の市場調査\cite{smartphoneresearch}によると、日本でのスマートフォン普及率は40\%前後と先進国の中ではやや低調である。元々高品質な携帯電話が普及しており、プラットフォームが盤石であったことが要因であると考えられている。}\\
一方、e-learningとは、パーソナルコンピュータなどの情報機器を用いて行う学習のことである。1990年代後半からのPCの普及と共に様々な分野で用いられるようになり、現在ではe-learningのコンテンツ共有を目的とした規格\cite{scorm}や大学設置基準に基づく文部科学省告示の中にe-learningに関する項目が記述される\cite{monkasho}など、制度や規格も整備されたものとなっている。\\
だがe-learningコンテンツを提供するサイトは、その多くがタブレット端末、スマートフォン端末が発表されるより前に製作されたものである。現状、スマートフォンやタブレットからe-learningコンテンツに対してダイレクトにアクセスするためには、まずはうまくコンテンツだけがヒットするような副次的な検索キーワードを考えて、PC用のサイトから小さなボタンをタップし、コンテンツをダウンロードし、更にテキストや動画で別々の検索エンジンを使わなければならない、といったようにかなりの労力を要する。\footnote{iOSについては、e-learning用にユーザーインターフェースが最適化されたiTunesU\cite{itunesu}が存在するが、これはiTunesStore内にあるコンテンツのみを対象としており、WWW上に存在するコンテンツをすべて検索対象とすることはできない}。\\
当研究では、こういった問題点を改良するため、スマートフォン端末・タブレット端末上でe-learningコンテンツを簡単に検索し、自在かつ直感的に閲覧、ダウンロードできるアプリの開発を行う。


\section{本文書の構成}
この第\ref{chap:introduction}章では、本論文を書くに至った背景とその構成を説明している。\\
第\ref{chap:webapi}章では、検索エンジンの構成に使用したWebAPIと、それらを統合した手法について説明する。第\ref{chap:search}章でキーワードを用いた精度向上手法について説明する。第\ref{chap:visualize}章では、WWW視覚化という目線での先行研究や開発事例、そして解決方法についての案を提起する。第\ref{chap:coding}章では、アプリ開発のための言語や手法についての詳細を概説する。第\ref{chap:ledoxea}章では、開発したアプリの利用方法と、その特徴について説明する。第\ref{chap:result}章では、そのアプリについての評価を行い、考察を述べる。第\ref{chap:conclusion}章では、本研究のまとめを行い、今後の課題を列挙する。	% 序論
\chapter{MashupとWebAPI}
\label{chap:webapi}
本章では、表題となっているMashupと呼ばれる開発手法に加え、WebAPIと呼ばれるタイプのAPIについて解説する。
\section{Mashupとは}
Mashupとは、2つ以上のWebAPIを組み合わせて1つのWebサービスやアプリケーションを構成する手法のことである。元来、利用価値の高いWebサービスを作るためには、独自に検索エンジンや結果応答用のサーバを構築する必要があり、目的のWebサービスを作るために多大な努力をする必要があった。しかし、Mashupでは、既存のWebサービスを組み合わせることにより、短期間で価値の高いWebサービスを製作することができる。
\section{WebAPIとは}
WebAPIとは、インターネットを介して利用することのできるアプリケーション・プログラミング・インターフェイス(API)のことである。殆どのWebAPIが一般的なURLの形式を取っており、HTTPによるPOSTメソッドを用いて、パラメータを付加したURLを使用してアクセスしてデータを取得する。返ってくるデータはXML、JSONのどちらかが一般的である。今回用いるWebAPIは、以下の4つである。
\subsection{Yahoo!検索Web API-ウェブ検索API\footnote{ウェブ検索APIは、2013年3月頃を目処にAPIのリクエストURLが変更される予定であり、これはそれまで公開されていたアップグレード版ウェブ検索APIを使用している。}}
\begin{tabular}{c|l}
開発 & ヤフー株式会社\\
URL & http://search.yahooapis.jp/PremiumWebSearchService/V1/webSearch\\
機能 & Web上に公開されているページを検索する
\end{tabular}
\subsection{Yahoo!検索Web API-画像検索API\footnote{画像検索APIは、2013年3月頃を目処にAPIのリクエストURLが変更される予定であり、これはそれまで公開されていたアップグレード版画像検索APIを使用している。}}
\begin{tabular}{c|l}
開発 & ヤフー株式会社\\
URL & http://search.yahooapis.jp/PremiumImageSearchService/V1/imageSearch\\
機能 & Web上に公開されている画像を検索する
\end{tabular}
\subsection{Youtube Data API}
\begin{tabular}{c|l}
開発 & Google Inc.\\
URL & http://gdata.youtube.com/feeds/api/videos\\
機能 & Youtubeの機能(動画の検索、アップロード、再生リストの作成など)を利用する
\end{tabular}
\subsection{Product Advertising API}
\begin{tabular}{c|l}
開発 & Amazon.com, Inc.\\
URL & http://ecs.amazonaws.jp/onca/xml\\
機能 & Amazon の商品情報や関連コンテンツを検索する
\end{tabular}	% WebAPIについて
\chapter{検索エンジンの精度向上}
\label{chap:search}

この章では、キーワードを用いた検索結果の精度向上手法について説明する。

\section{先行研究}
当研究室にて行われた研究として、以下の研究がある。

\subsection{e-learningコンテンツにおけるドキュメントサーチの最適化\cite{docsearch}}
Yahoo!Search BOSS APIを用いて、有用なWebページへのヒット率を向上させる実験が行われた。その結果、検索を行う主なキーワードとは別に、HTMLページ中に埋め込まれた<meta>タグのkeyword属性のパラメータとして特に多いものをサブキーワードとして検索を行う方法にて、有用なWebサイトへのヒット率が向上することが確認された。

\section{サブキーワードの選出}
上記研究結果から、e-learningコンテンツへのアクセス精度を高めるため、サブキーワードを選出する方法を考える。

\subsection{選出過程}
サブキーワードを選出する方法に上記の先行研究結果を用いることを試みたのだが、HTMLをキャッシュせず、検索を行うたびに毎回数十ページへアクセスを行なっていたため、タブレット端末やスマートフォン端末など、通信が不安定になる可能性が高い端末でこれを用いることは難しいと判断した。よって今回は、予めヒット率が向上すると考えられるキーワードを適当に予測、検証し、1〜2個程度のサブキーワードを選出した。

\subsection{選出結果}
\begin{itemize}
\item 基礎
\item 講座
\end{itemize}
以上の2個をサブキーワードとする。サブキーワードはORに設定して検索を行い、どちらか片方が、メインキーワードと共にヒットした場合の結果のみを引き出すよう設定する。この手法により、e-learningコンテンツのみに絞った検索結果をwebAPIより得られるよう構成した。	% 検索エンジンの精度向上
\chapter{WWW視覚化}
\label{chap:visualize}

本章では、本研究の背景、それを踏まえた上での研究の目標・目的、そして文書の構成について述べる。

\section{先行研究と開発事例}

\subsection{納豆ビュー}

\subsection{Flowser.com}


\section{検索エンジンにおけるWWW視覚化}

\subsection{Helix view}

\subsection{Star view}

\subsection{Star-Helix view}

\subsection{Star-Slide view}


\section{採用手法}	% WWW視覚化
\chapter{開発手法}
\label{chap:coding}

本章では、本研究の背景、それを踏まえた上での研究の目標・目的、そして文書の構成について述べる。

\section{概観}


\section{Away3D}

\subsection{Away3Dとは}

\subsection{Android上への移植}


	% 開発手法
\chapter{Androidアプリ「LEDOXEA」}
\label{chap:ledoxea}

本章では、本研究の背景、それを踏まえた上での研究の目標・目的、そして文書の構成について述べる。

\section{使用方法}


\section{特徴}

\subsection{移植性}

\subsection{スペック性能への非依存性}

\subsection{フリック操作による直感的操作性}

\subsection{検索エンジンの同時検索、WWW視覚化}


	% Androidアプリ「LEDOXEA」
\chapter{アンケートによる評価と考察}
\label{chap:result}

本章では、そのアプリについての評価を行い、考察を述べる。

\section{評価結果}
今回は、当研究室のメンバーに対してアプリを実際に利用してもらい、アンケートを行った。
その結果、以下の様な意見を得た。
\begin{itemize}
\item 非常に動きが独特で面白い。
\item 横や縦にスライドして動く描写が楽しく、e-learningコンテンツを見ていく面白さがあって良い。
\item パブリッシュされた一つのアプリとしてリリースするより、HTMLページとしてリリースする方が良い。より一般的に利用して貰えるようになるし、iOSやタブレット端末への移植の手間も省ける。
\item PDFファイルまでアクセスできるのは素晴らしいが、一度ダウンロードが行われてからビューワを起動するため手間がかかるのではないか。PDFファイルをHTMLで直接閲覧できるよう構築できるようにした方が良い。
\item タブレット等の横幅のあるデバイスならば、レイアウトを工夫して、画面遷移なしでブラウザを横に起動させたりできるのではないか。そうすればより良い使いやすさを追求できると考える。
\item 検索ボックスやタイトルの表示部が透過されているため、下のアイコンと被っており見にくい。
\item 検索ボックスが非常にシンプルなため、一見しただけではどのように使うのか分からない。説明を加えてほしい。
\item 現在どの結果に注目しているのかが分かりにくいため、アイコンを拡大したり、印をつけるなどしてほしい。
\item アイコンをタップすれば良いと思ってしまうが、実際には下のタイトル部分をタップしなければならないため、紛らわしい。アイコンから直接リンクできるようにすれば良いのではないか。
\item アイコンが大量に並ぶと、どれが重要なコンテンツなのか判断がつかない。表示数を制限して、より重要な結果に絞ってから表示するべきである。
\item 検索結果で出てきたアイコンは、単なる検索結果だけなので、その中における序列がなく最も有効なコンテンツがわかりづらい。検索結果を深く融合して、まとめてランク付けを行なってほしい。
\item URLからのアイコン画像取得に失敗するエラーがある。アプリ自体の価値が極度に下がってしまうため、なるべく改善した方が良い。
\item 画像ファイルの読み込みが遅い。画像処理専用のサーバ等を使い、ファイルをより早く処理できるようにすれば良いのではないか。
\item 全体的にどれが何を表しているのか分からづらい。ある程度慣れているユーザならば、PDFのマークには見覚えがあるが、初めて見たユーザや子供には、それが何を表しているのか、今ひとつ分からない。随所に説明を入れたり、アドバイス用のメッセージを出すなどしてほしい。
\item カスタマイズ性を重視して、拡張によって検索エンジンを導入できるような機能がほしい。
\item 表示形式を柔軟に変更できるようにしてほしい。画面を回転・ズームし、自分の好きな方向に簡単に移動できる機能があれば、より操作性を高めることができると考える。
\end{itemize}

\section{考察}
今回の解決すべき命題は、
\begin{itemize}
\item Mashupの特性を活かした、e-learningコンテンツ検索エンジン開発
\item Android端末上であることを意識した、ユーザビリティの高いWWW視覚化
\end{itemize}
の2点であった。

\subsection{e-learningコンテンツ検索エンジンとしての性能}
検索結果の一覧性が高いことや、PDFデータ等にダイレクトにアクセスできることを評価する声が一定数あった。この点より、当初の目的としての検索エンジン開発は概ね達成したと言えるだろう。その一方、APIを通じて得た検索結果が、アイコンとしてただ広がるだけのGUIに不満があったことは否めず、よりシンプルな、ひと目で分かる検索結果を返すための仕組みを、Mashupを通じて再度構成するべきだろう。

\subsection{Android端末におけるWWW視覚化とユーザビリティ}
スワイプ動作の独特な操作性や、レンダリングがスムーズで心地良く、面白いとの意見を多数得られた。この点より、Android端末としての利点は活かし、ユーザエクスペリエンスの点ではある程度の評価を得られたと考える。しかし、ユーザビリティの高さ、という部分が完全におざなりになっている面に関しては、意見の数々より自明のことであろう。また、納豆ビュー\cite{natto}に実装されていた、z軸方向における回転は実現できたものの、x-y方向への一意的な意味付け、回転操作等は実現できず、三次元空間における操作性の点に関しても、幾つかの課題が残った。より自由な操作性を実現するため、ローテートやダブルタップなど、より複雑な操作によって実現できることも検討するべきだろう。	% アンケートによる評価と考察
\chapter{結論}
\label{chap:conclusion}

ここまでの研究により、e-learningコンテンツをスマートフォン端末で検索するための三次元視覚化に対して、ある程度の成果と、今後の課題や方針を得られたと考える。それらを以下に示す。

\section{成果}
\begin{itemize}
\item Mashupによって、e-learningコンテンツを素早く、便利に検索するための手法を提案できた。
\item 納豆ビューを参考として、よりスマートフォン端末に最適化した形でのグラフ型WWW視覚化を発案できた。
\item Android端末上であることを意識し、ユーザにとって容易な操作方法を実装した。
\end{itemize}

\section{今後の課題と方針}
\begin{itemize}
\item 別のプラットフォームに対しての移植が可能であるかどうかを検討し、デバイス毎にチューニングを施していくことを考える。
\item それぞれの検索結果をより深く融合して、特徴空間におけるコンテンツの特徴ベクトルに基づくシステムへ再構成し、よりシンプルに結果まで辿り着くようにする。
\item ユーザが、より思ったとおりの操作を実現できるよう、メソッドや操作方法を追加・編集し、アプリとしてのユーザビリティを向上させる。
\item 画像ファイルの扱いをより極小化し、応答速度の向上を目指す。
\item 現在残っている数種のバグを取り除き、アプリとしての完成度をより高める。
\end{itemize}	% まとめ


\begin{acknowledgment}
本研究を卒業論文として完成させることができたのは、担当して頂いた新井康平教授、Herman Tolle博士研究員の熱心なご指導や、第4研究グループの皆様方に協力して頂いたおかげです。皆様へ心より感謝の気持ちと御礼を申し上げたく、謝辞に代えさせていただきます。
\end{acknowledgment}
	% 謝辞。要独自コマンド、include先参照のこと
\begin{bib}[100]

% \bibitem{参照用名称}
%   著者名: 
%   \newblock 文献名,
%   \newblock 書誌情報,出版年.

\bibitem{smartphoneresearch}
  株式会社シートプランニング:
  \newblock 世界のスマートフォン普及予測
  \newblock http://www.seedplanning.co.jp/press/2012/2012072601.html,2012年7月26日

\bibitem{hoge08}
  Taro Hogeyama, Jiro Hogeyama:
  \newblock The Theory of Hoge,
  \newblock {\it The Proceedings of The Hoge Society}, 2008.

\bibitem{scorm}
  Advenced Distributed Learning(ADL):
  \newblock SCORM,
  \newblock {\it http://www.adlnet.gov/capabilities/scorm}, 2004.
  
\bibitem{itunesu}
  Apple Inc.:
  \newblock iTunes U,
  \newblock {\it http://www.apple.com/jp/education/itunes-u/}, 2004. 
 
  \bibitem{monkasho}
  文部科学省:
  \newblock 平成十三年文部科学省告示第五十一号(大学設置基準第二十五条第二項の規定に基づく大学が履修させることができる授業等),
  \newblock {\it http://www.mext.go.jp/b\_menu/hakusho/nc/k20010330001/k20010330001.html}, 2001.

\end{bib}	% 参考文献。要独自コマンド、include先参照のこと
\appendix
\chapter{プログラム}
実装したActionScriptのソースコード、並びにAndroidアプリを定義するxmlファイルを掲載する。実装はFlashCS6を用いて、Android2.2端末(IS04)での動作を確認している。

\section{main.as(ActionScript)}
\begin{itembox}[l]{{未完成}}
{\scriptsize
\begin{verbatim}
ギリギリまで粘ったものを添付する予定です……
\end{verbatim}
 }
\end{itembox}

\section{app.xml(XML)}
{\scriptsize
\begin{verbatim}
<?xml version="1.0" encoding="utf-8" standalone="yes"?>
<application xmlns="http://ns.adobe.com/air/application/3.2">
  <id>legdoxea</id>
  <versionNumber>1.0.0</versionNumber>
  <filename>Legdoxea</filename>
  <description>3D view e-learning searcher</description>
  <name>Legdoxea</name>
  <copyright></copyright>
  <initialWindow>
    <content>Legdoxea.swf</content>
    <systemChrome>standard</systemChrome>
    <transparent>false</transparent>
    <visible>true</visible>
    <fullScreen>true</fullScreen>
    <autoOrients>false</autoOrients>
    <aspectRatio>portrait</aspectRatio>
    <renderMode>direct</renderMode>
    <depthAndStencil>true</depthAndStencil>
  </initialWindow>
  <customUpdateUI>false</customUpdateUI>
  <allowBrowserInvocation>false</allowBrowserInvocation>
  <icon></icon>
  <android>
    <manifestAdditions><![CDATA[<manifest>
      <uses-permission android:name="android.permission.INTERNET"/>
    </manifest>]]></manifestAdditions>
  </android>
  <versionLabel></versionLabel>
  <supportedLanguages>en ja</supportedLanguages>
</application>
\end{verbatim}
 }
		% 付録

\end{document}
