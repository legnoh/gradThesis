\chapter{結論}
\label{chap:conclusion}

ここまでの研究により、e-learningコンテンツをスマートフォン端末で検索するための三次元視覚化に対して、ある程度の成果と、今後の課題や方針を得られたと考える。それらを以下に示す。

\section{成果}
\begin{itemize}
\item Mashupによって、e-learningコンテンツを素早く、便利に検索するための手法を提案できた。
\item 納豆ビューを参考として、よりスマートフォン端末に最適化した形でのグラフ型WWW視覚化を発案できた。
\item Android端末上であることを意識し、ユーザにとって容易な操作方法を実装した。
\end{itemize}

\section{今後の課題と方針}
\begin{itemize}
\item 別のプラットフォームに対しての移植が可能であるかどうかを検討し、デバイス毎にチューニングを施していくことを考える。
\item それぞれの検索結果をより深く融合して、特徴空間におけるコンテンツの特徴ベクトルに基づくシステムへ再構成し、よりシンプルに結果まで辿り着くようにする。
\item ユーザが、より思ったとおりの操作を実現できるよう、メソッドや操作方法を追加・編集し、アプリとしてのユーザビリティを向上させる。
\item 画像ファイルの扱いをより極小化し、応答速度の向上を目指す。
\item 現在残っている数種のバグを取り除き、アプリとしての完成度をより高める。
\end{itemize}