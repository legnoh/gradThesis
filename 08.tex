\chapter{結論}
\label{chap:conclusion}

ここまでの研究により、e-learningコンテンツをスマートフォン端末で検索するための三次元視覚化に対して、ある程度の成果と、今後の方針を得られたと考える。それらを以下に示す。

\section{成果}
\begin{itemize}
\item Mashupによって、e-learningコンテンツを素早く、便利に検索するための手法を提案できた。
\item 納豆ビューを参考として、よりスマートフォン端末に最適化した形でのグラフ型WWW視覚化を発案できた。
\item Android端末上であることを意識し、ユーザにとって容易な操作方法を実装した。
\end{itemize}

\section{今後の課題}
\begin{itemize}
\item 別のプラットフォームに対しての移植が可能であるかどうかを検討し、デバイス毎にチューニングを施していくことを考える。
\item よりシンプルな検索結果を出せるよう、再度Mashupを活用し、シンプルな検索結果を提案できる手法を再考する。
\item ユーザが、より思ったとおりの操作を実現できるよう、メソッドや操作方法を追加・編集し、アプリとしてのユーザビリティを向上させる。
\item 画像ファイルの扱いをより極小化し、応答速度の向上を目指す。
\item 現在残っている数種のバグを取り除き、アプリとしての完成度をより高める。
\end{itemize}